\documentclass{article}
\usepackage{graphicx}
\usepackage{html}
\usepackage{hthtml}
\newcommand{\code}[1]{\texttt{#1}}
\usepackage{verbatim}

\begin{document}

\html{\begin{rawhtml}
    <h0>Basic References for ORA4MAS-NOPL</h0>
  \end{rawhtml}}
\latex{\begin{center}{\Huge 
     Basic References for ORA4MAS-NOPL}\end{center}}

%\maketitle

\html{\tableofcontents}

\section*{Introduction}

This document very briefly describes how to use the MOISE platform
based on CARTAGO. It follows the new approach to implement OMI
(Organisation Management Infrastructure): based on a normative
language and artifacts. The conceptual view of this approach is
presented in \cite{hubner:09e}, which is strongly based on ORA4MAS
\cite{hubner:09c}.

\subsection*{A brief history of MOISE OMIs}

\begin{itemize}
\item S-MOISE: a tier based centralised proposal. It adds an
  organisational layer on top of the communication layer. Agents use
  this new layer to interact/coordinate/.... It is centralised due a
  component called OrgManager that manages all the organisational
  layer \cite{hubner:05a}.
\item SYNAI: an extension of S-MOISE with context specification and
  more suitable normative specification \cite{gateau:07}.
\item J-MOISE (for S-MOISE): is the integration of S-MOISE for Jason
  agents \cite{hubner:07}. 
\item J-MOISE (for ORA4MAS): an integration of ORA4MAS with Jason
  similar to J-MOISE (for S-MOISE). This project was  discontinued,
  Jason agents access organisational artifacts as any other artifact.
\item ORA4MAS: the OMI based on artifacts. In the initial version the
  organisational artifacts are implemented in Java
  \cite{hubner:09c}. In a more recent version, the artifacts are
  programmed in NPL (Normative Programming Language)
  \cite{hubner:10b}.
\end{itemize}



\section*{Organisational Artifacts Specification}

Two kinds of artifacts are implemented: \code{GroupBoard} and
\code{SchemeBoard}. The usage interface (operations, observable
properties, and signals) of them are available in the following links
\begin{itemize}
\item \htlink{Group API}{../api/ora4mas/nopl/GroupBoard.html} and
\item \htlink{Scheme API}{../api/ora4mas/nopl/SchemeBoard.html} respectively.
\end{itemize}

\begin{rawhtml}
    <img src=ora4mas-arts-impl.png width=600/>
\end{rawhtml}

\section*{Java integration}

There is a generic GUI to create organisational artifact and agents
enabling the user to test the specification (it is similar to the
SimOE program in S-MOISE+). To run it:
\begin{verbatim}
ant run-art
\end{verbatim}

We can then set the organisational specification for the system and
create groups, schemes, and generic GUI based agents.

You can also read the code of a simulator to learn how to program Java
agents to use these artifacts. The source is available at
\htlink{src/ora4mas/nopl/simulator/ConsoleSimulator.java}{../../src/ora4mas/nopl/simulator/ConsoleSimulator.java}

\section*{Jason Integration}

\subsection*{Organisational actions}

The actions are the action provided by the OrgArts.

% When programming with Jason, the actions below are available. These
% actions are a simple mapping to the artifact operations (see thus the
% artifact operation documentation for more information).
% \begin{itemize}
% \item \code{create\_group(group instance id, os file, group type, has monitoring scheme, has gui)}
% \item \code{remove\_group(group instance id)}
% \item \code{create\_scheme(scheme instance id, os file, scheme type, has monitoring scheme, has gui)}
% \item \code{remove\_scheme(scheme instance id)}
% \item \code{add\_responsible\_group(scheme id, group id)}
% \item \code{adopt\_role(role, group id)} -- also focus on the group
% \item \code{leave\_role(role, group id)} 
% \item \code{commit\_mission(mission, scheme id)} -- also focus on the scheme
% \item \code{leave\_mission(mission, scheme id)}
% \item \code{goal\_achieved(goal, scheme id)}
% \item \code{set\_goal\_arg(goal, scheme id, argument, value)}
% \end{itemize}

\subsection*{Organisational perception}

Observable properties of the artifacts (which are Java objects) are
mapped to the beliefs inside agents by Cartago-Jason bridge. All of
them are annotated with \code{artifact(\textit{artifact id})} as shown
in the mind inspector:

\begin{rawhtml}
    <img src=org-bels.png />
\end{rawhtml}

The beliefs are the following (see API documentation of artifacts for more details):
\begin{itemize}
\item \code{specification(....)} for groups (see \url{../../doc/api/moise/os/ss/Group.html#getAsProlog()})
\item \code{specification(....)} for schemes (see \url{../../doc/api/moise/os/fs/Scheme.html#getAsProlog()})
\item \code{play(agent, role, group)}
%\item \code{responsible\_group(group, scheme)}
\item \code{commitment(agent, mission, scheme)}
\item \code{goalState(scheme, goal, list of committed agents, list of agent that achieved the goal, state of the goal)}
\item \code{obligation(agent,norm,goal,deadline)}: the current active obligations.
\item others (see observable properties of the artifacts)
\end{itemize}

The following signals are also translated to Jason events (see the
organisational artifacts specification for more details):
\begin{itemize}
\item \code{oblCreated(obligation)}: the state of the obligation is changed to ``created'' 
\item \code{oblFulfilled(obligation)}
\item \code{oblUnfulfilled(obligation)}
\item \code{oblInactive(obligation)}
\item \code{normFailure(details)}: some norm failure is thrown. 
\item \code{destroyed(artifact id)}: the artifact was destroyed.
\end{itemize}


\subsection*{Example}

We use the traditional \emph{writing paper} example to illustrate the
use of Jason - ORA4MAS integration \cite{hubner:09c}. In this example
we have three agents:
\begin{itemize}
\item Bob: will create one group, and two schemes to write two
  papers. He adopts editor in the group and mManager in the schemes.
\item Alice: this agent waits for the group creation and then adopts
  the role write and commits to the missions mColaborator and mBib.
\item Carol: this agent also waits for the group creation and then adopts
  the role write and commits to the mission mColaborator.
\end{itemize}

The Jason project is:

\verbatiminput{../../examples/writePaper/agents/writing-paper.mas2j}


% Note that Jason agents use a customised architecture to interact with
% the organisational artifacts. This architecture and some examples are
% available at
% \url{http://sourceforge.net/projects/jason/files/jason-tools}.


The code of the agents follows.
\subsubsection*{Bob}

\verbatiminput{../../examples/writePaper/agents/bob.asl}

The included file is:
\verbatiminput{../../examples/writePaper/agents/check_obl.asl}

\subsubsection*{Alice}
\verbatiminput{../../examples/writePaper/agents/alice.asl}

The common.asl file:
\verbatiminput{../../examples/writePaper/agents/common.asl}


\subsubsection*{Carol}
\verbatiminput{../../examples/writePaper/agents/carol.asl}

\bibliographystyle{plain}
\bibliography{jomi}


\end{document}